\documentclass{article}

\usepackage{amsmath}
\usepackage{amssymb}
\usepackage{amsfonts}
\usepackage[margin=1in]{geometry}
\usepackage{titling}
\usepackage{shellesc}
\usepackage{minted}
\usepackage[bookmarks]{hyperref}
\usepackage{xcolor}
\hypersetup{
    colorlinks,
    linkcolor={red},
    citecolor={blue!50!black},
    urlcolor={blue!80!black}
}

\setlength{\droptitle}{-7em}

\title{Supporting Calculations}
\author{Group 701: Tyler Tian, Julia Ye, Ivy Tan}

\begin{document}

\maketitle

\section{Source Code}

The code is split into two parts: A Python module that does all the calculations (\ref{sec:calc}), and another module
that provides a command-line interface (CLI) (\ref{sec:cli}).

Bridge designs are given to the program in YAML files. The specification for Design 0 (\ref{sec:design0}) and the final
design (\ref{sec:design}) are listed here.

The code depends on the following packages:
\begin{itemize}
    \setlength\itemsep{0em}
    \item \texttt{numpy}
    \item \texttt{matplotlib}
    \item \texttt{pyyaml}
    \item \texttt{click} (CLI only)
\end{itemize}

\subsection{Calculations Code}
\label{sec:calc}

Sample usage of the functions to do the calculations is included at the end of this code listing. All functions are
documented with docstrings, which includes example calls.

\inputminted[linenos, breaklines, fontsize=\small]{python}{calculate.py}
\pagebreak

\subsection{Command-Line Interface Code}
\label{sec:cli}

\inputminted[linenos, breaklines, fontsize=\small]{python}{bridgedesigner.py}
\pagebreak

\subsection{Bridge Specification Code}

\subsubsection{Design 0}
\label{sec:design0}

\inputminted[linenos, breaklines, fontsize=\small]{yaml}{design0.yaml}
\pagebreak

\subsubsection{Final Design}
\label{sec:design}

\inputminted[linenos, breaklines, fontsize=\small]{yaml}{new.yaml}
\pagebreak

\end{document}
